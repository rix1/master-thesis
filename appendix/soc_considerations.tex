\section{SoC Considerations} % (fold)
\label{sec:soc_considerations}

This section contain an overview over the different systems on a chip (SoC) and boards we considered.

RedBears Nano \cite{newRef:36} development boards were of particularly interesting because of their form factor. They are based on the Nordic NRF51822 chip and measure only 18.5mmx21.0mm. However, we because our goal was not related to mobility the actual size of the chip was not important to us. Another board of interest was Bitalino's development board for capturing body signals \cite{newRef:37}. Among other things, it comes with interfaces and built in analog to digital converter for ECG leads. This could have been a good starting point for our prototype, but the development kit only supports Bluetooth 2.0 which was of little interest to us.

We looked at Zolertia \cite{newRef:38} development boards, and even ran some preliminary tests using simulated Zolertia Z1 motes and the Contiki Cooja Simulator. However, these are based on the 802.15.4 specification, and was therefore outside the scope of this project. The same accounts for the Tmote Sky \cite{newRef:39}, which have been used extensively in earlier research \cite{Milenkovic:2006er, Owner:2006ub, ChulsungPark:2006tf, SteveWarren:2005ws, ChulsungPark:2006tf, Anonymous:GyP6wjY5}

Among more medical targeted development kits we looked at Shimmer Tools \cite{newRef:41} which offer professional, medical grade wireless sensors suited for ``academic, applied and clinical researchers''. These were of particular interest because of the configurable ECG options and that it shipped with libraries \cite{newRef:41} for streaming and parsing data on the Android platform. However, the Bluetooth module used for wireless transmission is based on the RN-42 chip \cite{newRef:43}, which only supports the 2.1 version of the Bluetooth protocol. Another, similar development kit targeting clinical trials, research and teaching labs, is the BioRadio from Great Lakes NeuroTechnologies \cite{newRef:44}. This is a highly configurable device with both adjustable sampling rate (250Hz-16kHz) and sample resolution (12-24 bit), as well as a SDK for pc \cite{newRef:45}. To our knowledge the BioRadio has a newer Bluetooth chip supporting version 4.0. However, because of the high sample rate and resolution, it is highly unlikely that they utilize the low energy features of the specification, due to the bandwidth constraints of the low energy specification. A further discussion on Bluetooth bandwidth versus the minimum requirements for ECG can be found in Sect. 3. It is also unlikely that the firmware is customizable in commercial products like the ShimmerTools and BioRadio, and was these two devices was therefore not interesting to us.

After investigating these different development platforms, we decided to avoid development kits capturing real physiological data (Bitalino, ShimmerTools and BioRadio) for the sake of testability and reproducibility. This way it would be easier to conduct consistent experiments. However, these devices offer a great deal of functionality and possibilities both in terms of hardware capabilities, technical specifications and ease of development, and should be considered in later research.

Texas Instruments another major actor in the embedded community. They design and develop a wide array of semiconductors and hardware for industries ranging from audio and hifi to medical devices \cite{newRef:46}. In fact they have a own sensory product line for biological signals. TI also has an active research community publishing reference designs for various applications branded under the name TIDesigns \cite{newRef:47}. In the later stages of this project, we became aware of an for a 5-lead ECG monitor based on Bluetooth Low Energy \cite{Anonymous:Q0qkTQkl}. The existence of this reference design have not influenced neither our design choices nor research questions, as it was discovered late in the project. An overview of the TI's proposed design and the implications an earlier discovery of this reference design could have had are discussed in Sect. 6.


% section soc_considerations (end)