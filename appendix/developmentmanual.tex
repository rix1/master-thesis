\section{Service Development Guide}
\label{devmanual}

This guide will handle the steps necessary to start developing custom services for the EMR Application.

To get started, create a new android project with a package named \textit{org.royrvik.emrservice}. Create a new file in the package and name it \textit{EMRRemoteInterface.aidl}. It is crucial that these exact package- and file names are used. Add the following code to the file:
\newline


\begin{lstlisting}[caption={EMRRemoteInterface.aidl}, label={lst:emrremoteinterface}]
// EMRRemoteInterface.aidl
package org.royrvik.emrservice;

interface EMRRemoteInterface {
    List<String> getPatientData(in String ssn, in String username, in String password);
    List<String> upload(in List<String> examinationData, in List<String> imagePaths, in List<String> notes, in String username, in String password);
}
\end{lstlisting}

\smallskip

Compile the newly created .aidl file, this will create a new Java interface in \textit{gen/org.royrvik.emrservice} within the project. It might be necessary to reopen the IDE for this file to appear. The setup of the AIDL is now complete.

To create the service itself, make a new class extending android.app.Service. In the service class, create a new \textit{private final EMRRemoteInterface.Stub} variable and initiate it with \textit{EMRRemoteInterface.Stub}. In the following example, the variable is named \textit{service}. The IDE should now complain that there are unimplemented methods in the Stub interface, let it auto implement the two methods \textit{getPatientData()} and \textit{upload()}. Now make the service's \textit{onBind()} method return the EMRRemoteInterface.Stub variable instead of the default \textit{null}. To make the service accessible from other projects and add \textit{android:exported="true"} to the \textit{service} tag in the \textit{AndroidManifest.xml}. From here, simply replace \textit{return null} in both implemented methods from the interface with custom code for the required environment. Following is an example implementation of the service and the manifest as well as a table of expected input/output.


\begin{lstlisting}[caption={ExampleService.java}, label={lst:exampleservice}]
package org.royrvik.emrservice;

import android.app.Service;
import android.content.Intent;
import android.os.IBinder;
import android.os.RemoteException;
import java.util.List;

public class ExampleService extends Service {
    public IBinder onBind(Intent intent) {
        return service;
    }

    private final EMRRemoteInterface.Stub service = new EMRRemoteInterface.Stub() {

        @Override
        public List<String> getPatientData(String ssn, String username, String password) throws RemoteException {
            //YOUR CODE HERE
            return null;
        }

        @Override
        public List<String> upload(List<String> examinationData, List<String> imagePaths, List<String> notes, String username, String password) throws RemoteException {
            //YOUR CODE HERE
            return null;
        }
    };
}
\end{lstlisting}



\newpage

\lstset{language=XML}
\begin{lstlisting}[caption={AndroidManifest.xml}, label={lst:manifest}]
<?xml version="1.0" encoding="utf-8"?>
<manifest xmlns:android="http://schemas.android.com/apk/res/android"
          package="org.royrvik.emrservice">
    <uses-sdk android:minSdkVersion="17"/>
    <application android:icon="@drawable/icon" android:label="@string/app_name">
        <service android:name=".ExampleService" android:exported="true"/> <!-- Important
    </application>
</manifest>
\end{lstlisting}

\newpage

\noindent Expected service input/output:
\newline
\textit{getPatientData()}
\begin{itemize}
\item Input:
    \begin{itemize}
    \item String ssn
    \item String username
    \item String password
    \end{itemize}
\item Output:
    \begin{itemize}
    \item List$<$String$>$
        \begin{itemize}
        \item 0 - int didWork
        \item 1 - String ssn
        \item 2 - String firstName
        \item 3 - String lastName
        \item 4 - String errorMesage
        \end{itemize}
    \end{itemize}
\end{itemize}
\newpage
\textit{upload()}
\begin{itemize}
\item Input:
    \begin{itemize}
    \item List$<$String$>$
        \begin{itemize}
        \item 0 - String ssn
        \item 1 - String firstName
        \item 2 - String lastName
        \item 3 - String examNumber
        \item 4 - String examTime
        \item 5 - String examComment
        \end{itemize}
    \item List$<$String$>$ imagePaths
    \item List$<$String$>$ notes
    \item String username
    \item String password
    \end{itemize}

\item Output:
    \begin{itemize}
    \item List$<$String$>$
        \begin{itemize}
        \item 0 - int didWork
        \item 1 - String errorMessage
        \end{itemize}
    \end{itemize}
\end{itemize}