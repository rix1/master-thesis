\section{Risk assessment}
\label{risk}

Probability: Prob.\\
\noindent Consequence: Cons.

\tablefirsthead{\hline \multicolumn{1}{|c}{\tbsp \textbf{Description}}
                       & \multicolumn{1}{c}{\textbf{Prob.}}
                       & \multicolumn{1}{c}{\textbf{Cons.}}
                       & \multicolumn{1}{c}{\textbf{Risk}}
                       & \multicolumn{1}{c}{\textbf{Preventive action}}
                       & \multicolumn{1}{c|}{\textbf{Remedial action}}\\ \hline\tbsp  }
\tablehead{\hline \multicolumn{6}{|l|}{\small\sl continued from previous page}\\
           \hline \multicolumn{1}{|c}{\tbsp \textbf{Description}}
                       & \multicolumn{1}{c}{\textbf{Prob.}}
                       & \multicolumn{1}{c}{\textbf{Cons.}}
                       & \multicolumn{1}{c}{\textbf{Risk}}
                       & \multicolumn{1}{c}{\textbf{Preventive action}}
                       & \multicolumn{1}{c|}{\textbf{Remedial action}}\\ \hline\tbsp  }
\tabletail{\hline\multicolumn{6}{|r|}{\small\sl continued on next page}\\\hline}
\tablelasttail{\hline}
\par
\captionof{table}{Risk assessment}

%\begin{adjustwidth}{-0.6in}{-0.5in}

\begin{supertabular}{|m{2.5cm}|m{1cm}|m{1cm}|m{1cm}|m{3cm}|m{3cm}|}
%\begin{supertabular}{|p{2.5cm}|p{1cm}|p{1cm}|p{1cm}|p{3cm}|{3cm}|}
\hline
\vspace{3 mm}
Failure to acquire the required information to implement the solution we want.\vspace{3 mm} & Med. & Med. & Med. & Be more aggressive with mails and phone calls to the people we are awaiting information from. & Find other solutions just to prove the concept. \\ \hline

Group member falls ill & High & Low & Med. & There is not much to do as a preventive action. & \vspace{3 mm}Redistribute workload on healthy group members. We consider this "lost time", and it is therefore not enough that the sick person catches up when healthy again. (Although this is important as well).\vspace{2 mm} \\ \hline

\vspace{3 mm}Information redundancy - writing and storing text/data on multiple platforms (Google drive, Confluence, ShareLaTeX). Other group members might not know what the newest file is.\vspace{2 mm} & Med. & Med. & Med. & Being strict on where we write down our info. & Set off time to reconstruct missing and/or outdated info. \\ \hline

Failing to satisfy/misunderstanding customer in regards to software requirements. & Low & Med. & Low & \vspace{3 mm}Continuos dialog with customer. An important part of agile development is to cope with change. We should therefore be prepared and ready for the situations this scenario will bring.\vspace{2 mm} & Discussion with customer what's wrong, and try to really understand what the customer wants. \\ \hline

Learning new tools (Confluence, Jira, Maven etc.) takes too much time. & Med. & Low & Low & \vspace{3 mm}Set of time to learn new tools. Realize that the start of the project will be slower than the end. Give different people responsible for learning different tools good, that way the "experts" can teach the others.\vspace{2 mm} & Ask for help! Involve other people in the learning process, avoid new pitfalls. \\ \hline

Bad group communication which leads to misunderstandings and slow progress. & Low & High & Med. & SCRUM stand-ups every weekday. Communication daily on workdays either in vitro or on Google+ Hangouts. & \vspace{3 mm}Realize we have a communication problem. Team building and getting to know each others strengths to help redistribute workload. Maybe bring the team together for something else than work.\vspace{2 mm} \\ \hline

Several group members fall ill & Med. & Med. & Med. & \vspace{3 mm}If one group member is sick, he should not feel obligated to attend or meet the team. Sickness spreads. The other team members must be prepared to work the amount of hours lost from the missing team member.\vspace{2 mm} & Redistribute workload.The other team members must take up the amount of work lost from the missing team member. Contact customer incase of we need to postpone a deadline. \\ \hline

Group members fall ill while another member is on leave/vacation. & Low & Med. & Med. & \vspace{3 mm}Use Team Calendar (a tool) to get an overview of when members are on leave. Do the preventive actions as planned for the previous scenario.\vspace{2 mm} & Redistribute workload. Contact customer incase of we need to postpone a deadline. \\ \hline

Misunderstandings lead to functionality not being implemented the way it were supposed to & Med. & High & High & \vspace{3 mm}This will lead to duplicate work. Sync every group member several times a week (in meetings) and ensure that JIRA issues is well described/commented."In doubt, there should be no doubt"\vspace{2 mm}. & If there is any doubt, ask the person who made the issue to give a better description of the task/functionality. It is better to postpone the implementation a couple of hours, than to work in a wrong direction. \\ \hline

Time pressure leads to shortcuts in communication. & Low & High & Med. & \vspace{3 mm}Plan the sprint well in the sprint planning meeting. Break functionality into small enough tasks that are well defined. Sort the issues after importance and ask questions whenever there is ambiguity about how to do something.\vspace{2 mm} & Drop a given issue if there is no time for it. It can wait for the next sprint. It is better to do things right and properly the first time, than to throw poor, semi-done functionality into a release. \\ \hline

SCRUM routines is not taken seriously and/or forgotten. & Med. & High & High & \vspace{3 mm}The group as a whole must tighten routines, and be more strict on each other to follow these. In the last couple of weeks we have seen that these routines can be the only glue holding the project together when goals and requirements change. This is important. \vspace{2 mm}& Meet. Discuss routines, and identify what parts/routines we have been missing out on. Set of more time to conduct these routines (sprint planning, daily scrum, backlog grooming, sprint retrospect etc). \\ \hline

Group member does not attend "Daily SCRUM" & High & Low & Med. & The reasons for not meetings can be many, therefore a single preventive action is not easy to determine. But the least thing we can do is to pick a time when nobody has any other activities, and let there be no misunderstanding about when/how we are doing this. & \vspace{3 mm}It is not enough that the missing attendant is writing a textual update on what he has been doing because he will still not have any clue about what the rest of the group has been up to. If one member is missing, we need to (1) document (textual) what everyone have been doing and share it among all parties. Or (2) move the Daily to a later time that day. \vspace{2 mm}\\ \hline

Group member does not attend work session & Low & Med. & Low & We consider this to be the same as "group member falls ill." & \vspace{3 mm}Redistribute workload on attending group members. We consider this "lost time", and it is therefore not enough that the missing person catches up next time, unless the person comes up with a precise plan on what to work on independently. Anyways, this implies that the group have to have a "Daily SCRUM" as described above.\vspace{2 mm} \\ \hline

Group member has to attend day-time job while the rest of the group has a work session. & High & Med. & High & Group member tries not to schedule day-time job while there's a work session. & \vspace{3 mm}If possible, work on project while on the job. Catch up on work in the evening, or day prior. Orientate group of situation well in beforehand.\vspace{2 mm} \\ \hline

\end{supertabular}

%\end{adjustwidth}


\subsection{Technical Risks}
\label{technicalrisks}
\begin{table}[H]
\renewcommand{\arraystretch}{1.2}
\captionof{table}{Technical Risks}
\begin{tabular}{|m{1.8cm}|m{3cm}|m{1cm}|m{1cm}|m{1cm}|m{4cm}|}
\hline
\textbf{ID} & \textbf{Description} & \textbf{Prob.} & \textbf{Cons.} & \textbf{Risk} & \textbf{Mitigation} \\ \hline
\textbf{Threat 1} & \textbf{Data leakage} &  &  &  &  \\ \hline
1.1 & Bypass authentication & M & H & H & - \\ \hline
1.2 & Exploit third party software & H & H & H & - \\ \hline
1.3 & Memory reading & M & H & H & - \\ \hline
1.4 & SQLCipher stack-trace & H & M & M & Wait for patch \\ \hline
\textbf{Threat 2} & \textbf{Backdoor} &  &  &  &  \\ \hline
2.1 & Code bug & M & H & H & Review code \\ \hline
2.2 & Misconfiguration & M & H & H & Educating technical users \\ \hline
\textbf{Threat 3} & \textbf{Stolen device} &  &  &  &  \\ \hline
3.1 & Guess user password & L & H & M & - \\ \hline
3.2 & Brute force login & H & H & H & Rely on LDAPS or implement max amount of tries on login \\ \hline
3.3 & Read database files & L & H & M & Encrypt database \\ \hline
3.4 & Access system from last session timeout & L & H & M & Keep a short session timeout. Educate user to log out. \\ \hline
\textbf{Threat 4} & \textbf{Social engineering} &  &  &  &  \\ \hline
4.1 & Bribe user & L & H & M & Educate users \\ \hline
4.2 & Bribe admin & L & L & L & Admin does not have access to data \\ \hline
4.3 & Acquire login for another user & L & L & L & User specific databases \\ \hline
\end{tabular}
\end{table}



