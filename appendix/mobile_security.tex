\section{Security on smart devices}
\label{smartSecurity}

The following section contains some examples on how vulnerable to attacks todays smart devices might be. 

\subsection{Smart devices}
Recently there has been an increase in research regarding the exploitation of so called side-channels on smart phones. In a larger context these channels conclude only some part of the whole attack surface, but by exemplifying this we might give the customer an idea of the wide array of possible threats. As Spreitzer describes in a recent paper \cite{2014arXiv1405.3760S}, these sensors poses a serious threat to the user's privacy and security. Side channels represent unintended information leakage during the operation of a device and potentially allow attackers to recover secret information. Spreitzer and others illustrates how one only by monitoring the built in sensors can measure variations in movement, acceleration, sound or even light intensity and successfully extract sensitive data like keystrokes or even RSA Keys \cite{cryptoeprint:2013:857}. These exploits would bypass software security layers like authentication and encryption, because they rely solely on physical factors.

\subsubsection{Memory}
Another problem we came across was how to handle in-memory data. Android is built on Java and the Dalvik Virtual Machine, where memory management is abstracted from the programmer. Instead it uses a Garbage Collector (GC) to reallocate memory that is no longer being used.
 
Through habits obtained from consumer products users expect a high degree of availability from modern mobile devices. In order to achieve this availability the devices are almost never powered off, and use several techniques like hibernation and sleep modes to preserve battery while keeping the device on. From a technical perspective this also results in keeping the main memory intact. Combined with the portability of mobile devices, this introduces several problems when managing sensitive data: encrypting stored data is no longer enough, as the main memory now is part of the attack surface. We fear the scenario where an attacker would just grab a device and walk away. As mentioned below, encrypting main memory (RAM) does not serve any real purpose, and it would put a lot of constraints on the performance of the device.
 
We found several techniques that makes in-memory data vulnerable as long as the attacker has physical access to the device. The Norwegian developed Inception tool \cite{inception} does exactly this. By exploiting I  EEE 1394 SBP-2 DMA, the software tool claims to unlock and escalate privileges to administrator/root on almost any powered-on machine you have physical access to. It does this by first gaining full read/write access to main memory through the DMA of a physical port. Then it searches the available memory pages, looking for the operating system's password authentication modules. When found, it modifies the code to allow any password. This works as long as the device is powered on, and the physical Direct Memory Access (DMA) compatible I/O ports on the device are enabled. Does this apply to the USB interface? While the lowest-level USB driver that talks to the root hub does DMA, we have yet to find out whether or not this exploit is possible via the USB interface (which is found on the Vscan), but it is not unthinkable. Another entrance for this exploit could be the SD card interface, while we have not researched this further.
 
Another, more advanced attack, that only requires physical access to the device is the \emph{Cold Boot Attack}. The attack is described in a paper by researches from Princeton University \cite{Halderman08leastwe}. This also exploits the data that resides in main-memory, but in a slightly different way. By cutting power to the device and cooling the RAM chip, the bits that reside in RAM fade away slowly. When this is done, one could remove the chip and read it via another device, or boot the device from an external source and copying the contents. This way disk encryption could also be broken because the encryption keys would reside in RAM as long as the device was powered on before the attack begun. Both these examples concludes in the fact that:

\begin{quote} 
\textit{Sensitive data in memory + physical access = major vulnerability}
\end{quote}

\subsection{Prevention}
In Real-Time Computing, making economical allocation of resources in scalable systems is an important problem. This problem can be solved with a programming style here where one is always concerned with the allocation and deallocation of resources. By manually deallocating in-memory data, you put less constrains on the GC and it gives you better control over what application data actually exists in the memory at a given time. A similar technique could be adapted in our case, in order to keep the exposure window of in-memory sensitive data to a minimum. This would have to be decided early in design phase, and is probably really expensive and time consuming to both develop and test.