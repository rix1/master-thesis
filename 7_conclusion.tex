\section{Conclusion} % (fold)
\label{sec:conclusion}

In this thesis we have 
\\
\newline
(...)
\\
\newline
A inexpensive, flexible and scalable low energy monitoring solution might enable more widespread collection of physiological data among patients outside the intended use presented in this thesis. Using available technology and open standards, this form of data collection used for patient monitoring can be achieved today. By using open web standards and standardized data formats like HL7 FHIR, one allow for increased information flow between hospital information systems, e.g. enabling separate data analysis systems and reasoning machines based on clinical knowledge bases to reason about the data. 

All together, this might contribute to the shift of today's reactive health care practices to a proactive one.

In conclusion this thesis has established some guidelines to some of the technical requirements of conducting wireless ECG. We've also managed to give some insight in the different practices of patient monitoring today, that hopefully can give some indications on what directions the future might take.

% section conclusion (end)