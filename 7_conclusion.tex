\section{Conclusion} % (fold)
\label{sec:conclusion}

In this thesis we have 
\\
\newline
(...)
\\
\newline
A inexpensive, flexible and scalable low energy monitoring solution might enable more widespread collection of physiological data among patients outside the intended use presented in this thesis. Using available technology and open standards, this form of data collection used for patient monitoring can be achieved today. By using open web standards and standardized data formats like HL7 FHIR, one allow for increased information flow between hospital information systems, e.g. enabling separate data analysis systems and reasoning machines based on clinical knowledge bases to reason about the data.

In conclusion this thesis has established some guidelines to some of the technical requirements of conducting wireless ECG. We've also managed to give some insight in the different practices of patient monitoring today, that hopefully can give some indications on what directions the future might take.

\begin{itemize}
  \item Summarize the discussion
  \item Answer research questions
  \item What does this mean?
\end{itemize}


\textit{In this master thesis we want to answer the following: Is it possible to create a monitoring solution for wireless ECG, based on available technology and open standards?}

Yes, this is possible and preferable as it could benefit . For practical, socio-technical and economic reasons, a multi-vendor WBAN environment seems improbable without special partnerships such as Apple's partnership with Epic. 

There is definitively a reason beyond the suppliers intention of making money. There is a demand 


% Looking at previous development in hospital information systems, ... there are economic factors that drives much of the innovation in health care ICT. In a sense, unless required by law, it can benefit a strong provider to refrain from standards. 

% section conclusion (end)