\section{Conclusion} % (fold)
\label{sec:conclusion}

In this thesis we have explored clinical wireless ECG monitoring.

% TODO oppsummere fra diskusjon
% TODO hva betyr dette?
% TODO svare på forskningsspm

A inexpensive, flexible and scalable low energy monitoring solution might enable more widespread collection of physiological data among patients outside the intended use presented in this thesis. Using available technology and open standards, this form of data collection used for patient monitoring can be achieved today. By using open web standards and standardized data formats like HL7 FHIR, one allow for increased information flow between hospital information systems, e.g. enabling separate data analysis systems and reasoning machines based on clinical knowledge bases to reason about the data.

In conclusion this thesis has established some guidelines to some of the technical requirements of conducting wireless ECG. We've also managed to give some insight in the different practices of patient monitoring today, that hopefully can give some indications on what directions the future might take.

We managed to create a prototype that performed well within the acceptable limits of end-to-end latency.

\textit{In this master thesis we want to answer the following: Is it possible to create a monitoring solution for wireless ECG, based on available technology and open standards?}

As we have discussed in previous sections, part of our underlying motivation for finding an answer to this question was to learn more about the future of sensor technology in a clinical setting. Within the context of our intended use, and on the attributes we wanted to evaluate,

indicates that it is indeed possible to wirelessly monitor ECG in a clinical setting using available technology and open standards.

As we have evaluated and established, using available technology and open standards it is indeed possible  to wireless  ECG using
 was to assess the feasibility of a multi-vendor sensor environment in a clinical setting.


As our conclusions Yes, this is possible and preferable as it could benefit . For practical, socio-technical and economic reasons, a multi-vendor WBAN environment seems improbable without special partnerships such as Apple's partnership with Epic. 

There is definitively a reason beyond the suppliers intention of making money. There is a demand 


% Looking at previous development in hospital information systems, ... there are economic factors that drives much of the innovation in health care ICT. In a sense, unless required by law, it can benefit a strong provider to refrain from standards. 

% section conclusion (end)