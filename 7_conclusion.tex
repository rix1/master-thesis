\newpage
\thispagestyle{empty}
This page has been intentionally left blank.

\newpage


\section{Conclusion} % (fold)
\label{sec:conclusion}

In this thesis we have explored clinical wireless ECG monitoring. Through qualitative research and the creation of a prototype based on a low energy wireless node, we have evaluated the technology against some aspects of wireless ECG. In this master thesis we want to answer the following: \textit{Is it possible to create a monitoring solution for wireless ECG, based on available technology and open standards?}

As we have discussed in previous sections, part of our underlying motivation for finding an answer to this question was to learn more about the future of sensor technology in a clinical setting. Within the context of our intended use, and on the attributes we evaluated, we have showed that it is indeed feasible to create such artifact today, using available and open technology such as Bluetooth Smart.

An inexpensive, flexible and scalable low energy monitoring solution might enable more widespread collection of physiological data among patients outside the intended use presented in this thesis. Using available technology and open standards, this form of data collection used for patient monitoring can be achieved today. By using open web standards and standardized data formats like HL7 FHIR, one might allow for increased data flow between hospital information systems and personal sensory equipment.

In conclusion, this thesis has established some guidelines to certain technical requirements of conducting wireless ECG. This was done through the creation of a prototype that performed well within the acceptable limits of end-to-end latency. We have highlighted different clinical considerations that are an integral part of doing research within this field. We've also managed to give some insight in the different practices of patient monitoring today, that hopefully can give some indications on what directions the future might take. Although using open and standardized technology could enable a multi-manufacturer organization of WBANs in hospitals today, we don't believe these solutions will see the light of day before both patients and clinicians alike place such a demand.

% section conclusion (end)