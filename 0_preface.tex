	\noindent \textbf{Preface}
  \newline
  	
\noindent  
This thesis is the result of a collaboration between the Department of Computer and Information Science (IDI) and Department of Telematics (IT) at NTNU and was conducted during the fall and spring semester 2015/2016.

For this research project we originally wanted to investigate how feasible it was to introduce todays commercial wireless low-energy technology in a medical environment. What were the shortcomings? Where are the barriers? What are the requirements? As discussed in Section~\ref{sec:method}, our original approach to answering these questions was through the development of an prototype that solved the practical problem of ambulatory patient monitoring.

Due to the multidisciplinary nature of the topic, the focus has shifted and been changed along the way. In the end we have evaluated the low energy features of Bluetooth Smart on a set of requirements using a mix of available tactics and research methods.

This project was supervised by Prof. Pieter Jelle Toussaint from the Department of Computer and Information Science and Associate Professor Frank Alexander Kraemer form the Department of Telematics at NTNU.