\begin{preface}

\section{Problem Statement}
Electronic patient monitoring has been practiced at hospitals for almost half a century [CITE][patient_monitoring_history]. Metrics like ECG, arterial blood pressure, SpO2, respiratory rate and body temperature are measured on every patient in a modern intensive care unit. The consumer industry, the driving force behind many technological advances, have the last couple of years also showed interest in capturing physiological data. With this interest comes new ideas, new solutions and new technology from companies that embed a customer centric focus. The health care has over the years grown into a collaboration of technologically isolated silos, whereas the consumer industry is working towards a ubiquitous reality [CITE][mark_weiser_the_computer_for_the_21st_century]. Here, every device and service is connected to each other, and available at any given time. Having the ability to monitor several physiological metrics wirelessly together holds potential to reduce cables and ease everyday routines, as well as giving ambulatory patients the freedom to move. However, there is a wide array of different measuring techniques and wireless protocols that all have different capabilities with inherent strengths and weaknesses. Further, the technical requirements to each measurement may vary according to the use case and patient's condition. This is a big obstacle in the implementation of such monitoring systems.

The purpose of this master thesis is to investigate the required capabilities of a wireless monitoring system, in order to assess the feasibility of introducing low energy wireless sensors for data collection. We will investigate how these capability requirements can be addressed experimentally for a set of use cases.

\section{Abstract}
In order to build an artifact that enables wireless monitoring of ambulatory patients, there is a need to understand both the domain and technology involved. In this thesis we investigated and assessed the required capabilities of a wireless monitoring system with the intention of creating a reference model for later projects. 

In the thesis we evaluate a proposed architecture for wireless patient monitoring implemented with Bluetooth Low Energy.

\section{Preface}
This thesis is the result of a collaboration between the Department of Computer and Information Science (IDI) and Department of Telematics (IT) at NTNU and was conducted during the fall and spring semester 2015/2016.

For this research project we originally wanted to investigate how feasible it is to introduce todays commercial wireless low-energy technology in a medical environment. What were the shortcomings? Where are the barriers? What are the requirements? As discussed in section 2, our original approach to answering these questions was through the development of an artifact that solved the practical problem of ambulatory patient monitoring.

Due to the multidisciplinary nature of the problem statement, the focus has shifted and been changed along the way, which resulted in the need for a more general problem statement.

\end{preface}