\section{Discussion} % (fold)
\label{sec:discussion}

In this chapter we will pull together our findings from Chapter~\ref{sec:technology_assessment} and~\ref{sec:clinical_context} and discuss alternative approaches, strengths and weaknesses with this research project. 

We believe the usage of smart devices in health care will increase in the coming years, and that the amount of cables will decrease. A ubiquitous future where both medical staff and patients use their private or hospital provided smart/touch devices as a tools in treatment and everyday medical related tasks, seems probable in a not too distant future. Some forms of this have already been deployed, like AFGA Healthcare's ORBISme! tablet sized EMR-device, being used in Germany today \cite{newRef:271}. 

There is however an important difference between consumer technology and medical devices. But with the growing interest for personal health monitoring within the consumer market, this distinction easily becomes blurry. Although not having any diagnostic value, it is not improbable that consumer devices soon will see it's way into a medical setting one way or the other\footnote{ Just recently was the first reported case of medical professionals using historic data form a personal fitness tracker to make an informed, life saving medical decision \cite{newRef:29}.}.

In the Section~\ref{sec:technology_assessment} we presented a solution where patient's touch devices was used as a WBAN hub, a mobile gateway connecting WBAN sensors to local area network and external services. On the same side, the consumer market has recently showed an increased interest in ``smart access points'' or ``IoT hubs''. These are WiFi access points with built in support for Bluetooth Smart and 802.15.4 in addition to the standard 802.11.X interface\footnote{ For more information, see OnHub WiFi router from Google \cite{newRef:60} and Cassia Hub Bluetooth Router \cite{newRef:61}.}. Deciding on whether the sensor devices should communicate directly with a smart access point, or through a personal gateway, is an important discussion as pointed out in the paper ``The Internet of Things Has a Gateway Problem'' by Zachariah et. al.~\cite{Zachariah:2015cm}. in the following section we will discuss the possibilities and restrictions with our chosen architecture.

\subsection{Monitoring Architecture} % (fold)
\label{sub:monitoring_architecture}

In Section~\ref{sec:clinical_context} we saw how single-manufacturer, specialized monitoring solutions are deployed at two Norwegian hospitals today. We recognize that this solution does not afford clinical WBAN support, and propose that tomorrow's clinical environment will be characterized by widespread use of sensory technology delivered by different providers and manufacturers.

For our proposed design we chose a two-tier architecture based on a personal\footnote{ Personal in the sense that each patient has it's own gateway. Whether or not these are private devices or hostpial provided raises several security and privacy questions which is the topic for another discussion entirely.}, mobile gateway. To enable a multi-manufacturer sensor environment and ensure interoperability in a scalable manner, we proposed using a repository pattern where a mobile gateway could install the ``profiles'' required for mapping sensor data to the correct formats based on the \emph{context}, i.e. what sensors was connected, what is the patient's condition, and at what unit in what hostpial is the patient located in. 

However there are tradeoffs with this solution. Making a single gateway the sink of several sensors organized in a WBAN creates a single point of failure, as well as a bottleneck. As mentioned in Section~\ref{sub:wireless_body_area_networks}, previous research have proposed categorizing WBAN data in different tiers, based on clinical priority. A real-time ECG stream may be of higher clinical value and have higher QoS requirements than the occasional temperature measurement\footnote{ Note that deep Vein Thrombosis, or blood clots, can be detected by sudden temperature changes in extremities.}.  With a single point of failure, ensuring the personal gateway is working properly will become of outmost importance, as a faulty gateway will not only take down the low priority sensor measurements, but also the critical ones. It is worth noting that today's telemetry solutions operate with zero downtime according to one of the clinical engineers we talked to.

As suggested by Shahamabadi et. al.~\cite{Shahamabadi:2013df}, this architecture is also susceptible to creating a bottleneck. With a growing number of sensor nodes participating in the WBAN, the amount of data being routed through the gateway increases, and the available capacity for intermediate analysis or other processing such as data transformation (as suggested in Section~\ref{sub:system_architecture}) decreases. 

An alternate approach involve using mobile gateway just as a facilitator for ``administrative'' tasks like coordinating access point handoff and roaming on behalf of the WBAN. This solution require that the WBAN sensors send physiological data directly to a smart access point. This proposition solves the problem of the mobile gateway becoming a bottleneck because data does not have to be routed through the gateway. A smart access point would also have more resources for doing intermediate data-processing~\cite{DrAmirMohammadRahmani:2014vx}. However, there is no silver bullet, as this has approach involve a higher energy consumption: With this organization the node antennas would have to consume more energy in order to send data directly to an access point/boarder router which would typically be located further away than the mobile gateway. 

We would also suggest this reduces the flexibility and interoperability of he monitoring solution. Should the backend monitoring system not be developed by the same manufacturer as the WBAN nodes, data need to be transformed into compliant application layer data formats for the information exchange\footnote{ This should indeed be standardized between manufacturers and wireless protocols, but from the experience of HL7 and OpenEHR we know this to be a hard problem.}. From a practical standpoint, we believe this transformation and similar pre-processing of data scales better at a personal gateway tier. Based on the assumption that the environment will be crowded by similar sensors from different manufacturers, we believe it is easier for personal gateways to maintain this responsibility independently, rather than all access points supporting all sensors present in the clinical setting.

% subsection monitoring_architecture (end)

\subsection{Research methods and results} % (fold)
\label{ssub:research_methods_and_results}

% TODO Network disadvantage - we had no control over the network traffic

% TODO battery of gateway does also play an important role

We need a section where we discuss the Bluetooth throughput.

Had we investigated the limitations of Bluetooth earlier in the project, we might have - as we first discovered this when implementing the code.

Gateway: In retrospect, we see that aiming for making a flexible solution took much more time to develop, and proved to be generate more edge cases, making the experimentation more prone to errors. Instead we could have hard-code   d the devices and endpoints into the gateway, which could have eased the execution of the end-to-end experiment. We made a mistake by prioritizing functionality over reliability in the gateway implementation.



On interviews: An alternative approach could have been to target departments higher in the organizational level like purchasing organizations, and do the comparison on the information gathered. This approach however could easily suffer from a mismatch between the general guidelines and the deployed technology as well as long turnover times, meaning a lot of time would have gone by just waiting for an answer from the respective higher level organizational departments.

Temporary list of limitations of our study:

\begin{itemize}

  \item We have not looked into signal nor data compression.

\end{itemize}

During the interview with one of the medical professionals it was mentioned that in his experience, the prevalence of cardiac abnormalities of medical interest might happen during physically straining activities like walking stairs etc. However, today they had no practical way of seeing this correlation. The only place where you can monitor the real-time ECG stream is at the central monitoring station. A great deal of planning and synchronization would have to happen in order to conduct a simple ECG-test on a patient under physically strain: One would have to record the time and place while watching the patient doing the activity, then return to the monitoring central and look at the history in order to get a clear view of how the patient's heart reacted to the activity. A better solution would be ability to watch the ECG on a tablet device while observing the patient doing the activity. An improved version of this would be to record indoor location, a step counter and altitude together with the ECG. This way, the medical professional wouldn't have to consult with neither the patient nor the monitoring central. They could just look at the historic ECG data with activity data overlaid.

% subsubsection research_methods_and_results (end)

\subsubsection{Future work} % (fold)
\label{ssub:future_work}

From a priori knowledge and studying Bluetooth Smart we know that the biggest energy consumer is the radio during the actual wireless transmission. In essence, version 4.0 of Bluetooth were able to reduce the energy consumption drastically compared to previous versions by powering off the radio in between events, instead of keeping a persistent connection. Based on this and our insight in today's practices we realize that there might be worth investigating radically different monitoring practices - like self-monitoring sensors. There is a trade off between doing onboard analysis on the sensor chip versus streaming the data continuously to a WBAN gateway. It might be more energy efficient to implement algorithms for analyzing sampled data on the sensors themselves and only transmit data on demand from clinicians or when deviations from given thresholds occur.

Should small, inexpensive and available patient monitoring solutions enable monitoring of every patient, regardless of medical condition, problems regarding data storage and compression will have to be addressed. Our calculations from Section~\ref{ssub:throughput} show that collecting raw ECG data from a single patient amount to between 388,8 MByte\footnote{3 leads, 24 bit sample size and 500 Hz sampling rate} and 3110,4 MByte\footnote{3 leads, 64 bit sample size and 1500 Hz sampling rate} on a daily basis. Yearly this amounts to a Terabyte of data per patient on a single physiological metric.

% subsubsection future_work (end)

% section discussion (end)