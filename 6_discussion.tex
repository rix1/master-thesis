\section{Discussion} % (fold)
\label{sec:discussion}

In this chapter we will pull together and discuss our findings from Chapter~\ref{sec:clinical_context} and~\ref{sec:technology_assessment}. Further we will discuss alternative approaches as well as strengths and weaknesses with our research project. 

We believe the usage of smart devices in health care will increase in the coming years, and that the amount of cables will decrease. A ubiquitous future where both medical staff and patients use their private or hospital provided touch devices as a tools in treatment and everyday medical related tasks, seems probable in a not too distant future. Some forms of this have already been deployed, like AFGA Healthcare's ORBISme! tablet sized EMR-device, being used in Germany today \cite{newRef:271}. 

There is however an important difference between consumer technology and medical devices. With the growing interest for personal health monitoring within the consumer market, this distinction easily becomes blurry. Although not having any diagnostic value, it is not improbable that consumer devices soon will see it's way into a medical setting one way or the other.\footnote{ Just recently was the first reported case of medical professionals using historic data form a personal fitness tracker to make an informed, life saving medical decision \cite{newRef:29}.}

In the Section~\ref{sec:technology_assessment} we presented a solution where patient's touch devices was used as a WBAN hub, a mobile gateway connecting WBAN sensors to local area network and external services. On the same side, the consumer market has recently showed an increased interest in ``smart access points'' or ``IoT hubs''. These are WiFi access points with built in support for Bluetooth Smart and 802.15.4 in addition to the standard 802.11.X interface.\footnote{ For more information, see OnHub WiFi router from Google \cite{newRef:60} and Cassia Hub Bluetooth Router \cite{newRef:61}.} Deciding on whether the sensor devices should communicate directly with a smart access point, or through a personal gateway, is an important discussion as pointed out in the paper ``The Internet of Things Has a Gateway Problem'' by Zachariah et. al.~\cite{Zachariah:2015cm}. in the following section we will discuss the possibilities and restrictions with our chosen architecture.

\subsection{Toward a Multi-Vendor Situation?} % (fold)
\label{sub:toward_a_multi_vendor_situation}

% subsection toward_a_multi_vendor_situation (end)
In Section~\ref{sec:clinical_context} we saw how single-manufacturer, specialized monitoring solutions are deployed at two Norwegian hospitals today. 

% In a opt this solution does not afford clinical WBAN support, and propose that tomorrow's clinical environment will be characterized by widespread use of sensory technology delivered by different providers and manufacturers.

For our proposed design we chose a two-tier architecture based on a personal\footnote{ Personal in the sense that each patient has it's own gateway. Whether or not these are private devices or hospital provided raises several security and privacy questions which is the topic for another discussion entirely.}, mobile gateway. To enable a multi-manufacturer sensor environment and ensure interoperability in a scalable manner, we proposed using a repository pattern where a mobile gateway could install the ``profiles'' required for mapping sensor data to the correct formats based on the \emph{context}.

We would also suggest this reduces the flexibility and interoperability of he monitoring solution. Should the backend monitoring system not be developed by the same manufacturer as the WBAN nodes, data need to be transformed into compliant application layer data formats for the information exchange.\footnote{ This should indeed be standardized between manufacturers and wireless protocols, but from the experience of HL7 and OpenEHR we know this to be a hard problem.} From a practical standpoint, we believe this transformation and similar pre-processing of data scales better at a personal gateway tier. Based on the assumption that the environment will be crowded by similar sensors from different manufacturers, as proposed by~\cite{DrAmirMohammadRahmani:2014vx}, we believe it is easier for personal gateways to maintain this responsibility independently, rather than all access points supporting all sensors present in the clinical setting.

However, this is not an easy assumption to be made, as there are both economic, security, maintenance and performance concerns that drive the implementation that we see at todays hospitals. Unless larger partner ships between different companies are established (as we saw with Apple in 2014~\cite{WhyAp90:online}), it is unlikely that any manufacturer will come forward and develop a platform or infrastructure for supporting WBANs. Taking Philips' IntelliVue telemetry solution as an example, it just isn't profitable for them to open up for other suppliers to make use of the infrastructure they have installed at hospitals around the globe. That being said, this situation is similar to the one we are experiencing everyday through our internet connections -- so to rule out that a change like this might happen, is unwise. However, we believe that as long as wireless sensors and the idea of a ``connected self'', is limited to wristbands and bathroom scales - we won't see a change in the hospital infrastructure in regards to supporting multi-manufacturer WBANs. 

% subsection monitoring_architecture (end)

\subsection{Research methods} % (fold)
\label{ssub:research_methods}

Working with this many different components and across several domains have also proven to be a challenge. As mentioned in the introduction, scoping this project had to be done along the way. Because the goals and intentions also shifted, we ended up making decisions that had to be revisited and in some cases undone at a later point.

Initially we were interested in the practical organization of a future, ubiquitous and distributed monitoring solution based on personal touch devices. We asked questions like ``How could alarm triggering events be distributed to different clinicians not located at a central monitoring station?'', ``How can clinicians be assured that patient sensors have enough battery for registering an important event? How does this scale?''. However, these questions was in them self based on assumptions about systems that does not yet exist. Therefore we saw the need for taking a step back and investigate todays existing practices, and the more fundamental aspects of ECG and wireless technologies. We recognize that the practical organization and structure of a wireless and ubiquitous monitoring solution might be a relevant problem statement in the future.

After taking a step back, we realized two things: A multi-manufacturer environment is not well supported by todays hospital infrastructure. This is a fundamental requirement in a future where patients in a larger degree capture sensing information about their bodies outside the hospital. 

And from previous literature very few define a clear link between the research and the indented use. As we learned, establishing this intended use is necessary in order to say anything about minimum requirements like throughput, and end-to-end latency. And as we discovered, even when having established a intended use, deciding on these metrics are still no easy task.

After having set the path we did, we are satisfied with the decision of comparing two different hospitals. As we visited the second hospital, several misconceptions were cleared and we got a broadened view, especially on the organization of patients and how the the practice of on-demand monitoring were conducted in relation to the telemetry monitoring. The challenge with qualitative research methods however, is to convert increased personal insight and experience, into knowledge for everyone else. We took an explanatory approach to this in Chapter~\ref{sec:clinical_context}, where we focused on explaining the systems and practices in a logical manner. We did not structure the chapter according to the source of the information, and rather focused on mixing sources. Hopefully this created a more compound impression of today's systems and practices.

We spent a lot of time finding the minimum requirements for ECG, without ever finding any gold standards. This is why we present our own guidelines to this in Chapter~\ref{sec:technology_assessment}. Although we searched literature, the reason we didn't find this might be related to our approach and research strategy. For the technical aspects of our prototype, the heavy focus on practice and use cases did not play as big of a role as we first imagined it to do. An alternative approach could have been to target departments and instances at a higher organizational level for our interviews. Here access to standardization documents and practices might be more accessible, as compliance is usually the concern of middle, and higher level management. However, this approach however could easily suffered from a mismatch between the general guidelines and the technology deployed in practice. This would also increase turnover times, meaning a lot of time would have gone by waiting for answers from the respective higher level organizational departments.

In terms of our data collection, one thing is clear: Getting information from the manufacturers and solution providers is not worth pursuing unless a collaboration or sponsorship is established before the research begins. The time spent running down dead ends in order to find ECG related technical specifications on commercial solutions, was in the end deemed unproductive.

In retrospect we acknowledge that ISO Standards probably would have given us a good insight on the matter, especially ISO 11073-91064:2009 and ISO/IEEE 11073-10406:2012.


% subsubsection research_methods (end)

\subsection{Implementation} % (fold)
\label{sub:implementation}

% TODO Why did we not implement the proposed repository pattern?

In retrospect, we see that aiming for making a flexible solution took much more time to develop, and proved to be generate more edge cases, making the experimentation more prone to errors. Because of the complex setup of our end-to-end latency experiment, more time could have gone into planning this making it more robust. In terms of the gateway, we could have focused less on the user interface. One alternative approach could have been to hard-code the devices, characteristics and endpoints into the gateway, which could have eased the implementation, debugging and testing. 

By the time were conducting the first trails of the experiment, we saw that the code (especially on the gateway) did not perform well over time, which was critical for our experiment. We had to do modifications on the fly during the experimentation phase in order to make everything work. We unit tested the individual components manually as we developed, but we left little time for integration tests before conducting the experiment. Had this been planned better we would have reserved time for integration tests. All in all, we see that testing and debugging the time synchronizing on both clients, as well as working with the control flow and user interface of the gateway proved to be most time consuming. It should be mentioned that we intentionally spent as little time as possible on writing custom code for the sensor board. This was because we had little experience with hardware and software development for low energy development kits. 

Implementation wise we admit more time should have been put into prioritizing reliability over functionality.

% subsection implementation (end)

\subsection{Results} % (fold)
\label{sub:results}

% subsection results (end)

\begin{itemize}

  \item [] \textbf{Throughput:} We decided to exclude signal and data compression from our scope. Investigating this would have made an impact on both the required throughput and the battery usage. Other than this, we hope to have established a good understanding of the different considerations that go into wireless ECG. It is worth noting that we compared our solution to Philips' EASI telemetry lead setup, which uses 3 leads to derive the rest. As such our results illustrates the minimum required throughput (without considering protocol overhead) for a ECG setup measuring 3 actual leads. In another use case, where the intended use is different, the number of actual leads would have to be multiplied accordingly.
  \item [] \textbf{End-to-end latency:} As the results from our experiment showed, the end-to-end latency is within the acceptable limits. These limits were however set based on only one related study. They clinically tested latency in a ECG monitoring system, but their intended use was in a \emph{telemedicine setting}. It should therefore be noted that the requirements in a in-hospital setting might be different.
  
  A weakness in our experiment was that it was only conducted in accordance to \texttt{U1}, our first use case. We therefore have little insight in how our prototype would perform in a semi-ambulatory or ambulatory setting. Conducting the experiment with the other use cases would have required another experiment setup. Not only because of the USB cable used to decide when values was sent from the sensor node, but also because the increased variables \emph{movement} introduces. This variability will require more details about the communication in-between the different components in the end-to-end test. 
  
  Our approach of testing the delay in this ``integration testing'' type of manner, proved to be both time-consuming, error prone and gave few answers to exactly \emph{why} the delay was as measured. An example of the lack of details can be illustrated with the spikes we experienced. With our setup we had no way of investigating these spikes without doing larger changes in the experiment setup or the code. However, because our desired goal was measured in the scale of \emph{seconds}, we expected our results to only give us an indication of the feasibility of our prototype. Because results was well within the established limits and the experiment measured delay at the application layer of the network stack, we know the results can be optimized further in the future. Based on this we were satisfied with the experiment and it's results.

  \item [] \textbf{Battery:} In our evaluation of the battery capacity we excluded investigating the battery consumed when sampling and processing data. This decision was based on the assumption that the radio would be the largest consumer of energy in our setup. However, this assumption makes our already general approximates even more uncertain, and the battery consumption ECG streaming devices should be investigated further. In Section~\ref{sub:future_work} we recommend an alternative approach that may reduce the radio usage further. Another thing we did not assess in our evaluation was the personal gateway's battery. Although one fundamental requirement of the gateway will be that it affords recharging, the actual battery consumption of continually (and periodically depending on the situation) streaming is another factor that threaten the feasibility of our proposed design.

\end{itemize}


% subsubsection results (end)


\subsection{Future work} % (fold)
\label{sub:future_work}

% subsection future_work (end)

From a priori knowledge and studying Bluetooth Smart we know that the biggest energy consumer is the radio during the actual wireless transmission. In essence, version 4.0 of Bluetooth were able to reduce the energy consumption drastically compared to previous versions by powering off the radio in between events, instead of keeping a persistent connection. Based on this and our insight in today's practices we realize that there might be worth investigating radically different monitoring practices - like self-monitoring sensors. There is a trade off between doing onboard analysis on the sensor chip versus streaming the data continuously to a WBAN gateway which have to be investigated further. It might be more energy efficient to implement algorithms for analyzing sampled data on the sensors themselves and only transmit data on demand from clinicians or when deviations from given thresholds occur.

Exploring the boundaries of classical monitoring is exiting, and might be proven to yield results: During one interview with one of the medical professionals it was mentioned that in his experience, the prevalence of cardiac abnormalities of medical interest might happen during physically straining activities like walking stairs etc. However, today they had no practical way of seeing this correlation. The only place where you can monitor the real-time ECG stream is at the central monitoring station. A great deal of planning and synchronization would have to happen in order to conduct a simple ECG-test on a patient under physically strain: One would have to record the time and place while watching the patient doing the activity, then return to the monitoring central and look at the history in order to get a clear view of how the patient's heart reacted to the activity. A better solution would be ability to watch the ECG on a tablet device while observing the patient doing the activity. An improved version of this could record indoor location, number of steps and altitude together with the ECG. This way, the medical professional wouldn't have to consult with neither the patient nor the monitoring central. They could just look at the historic ECG data with activity data overlaid.

Should small, inexpensive and available patient monitoring solutions enable monitoring of every patient, regardless of medical condition, problems regarding data storage and compression will have to be addressed. Our calculations from Section~\ref{ssub:throughput} show that collecting raw ECG data from a single patient amount to between 388,8 megabyte\footnote{3 leads, 24 bit sample size and 500 Hz sampling rate} and 3110,4 megabyte\footnote{3 leads, 64 bit sample size and 1500 Hz sampling rate} on a daily basis. This is only raw data, and packaged in a standard compliant data format, this data would occupy a lot of storage. Yearly, only the raw data amounts to a Terabyte of data per patient on a single physiological metric.

One key area that is in need for inexpensive solutions based on available technology and open standards are within ``Information and Communication Technologies for Development'' or ICT4D for short. This denotes using information and communication technology in the fields of socioeconomic development, international development, and human rights. Enabling inexpensive ECG monitoring in a rural area is definitively a possible direction for further development of this technology.

% subsubsection future_work (end)

% section discussion (end)