\section{Discussion} % (fold)
\label{sec:discussion}

Intro.


  % textit{What are the main challenges of introducing a WBAN monitoring system in clinical environments?}

\subsection{Architecture} % (fold)
\label{sub:architecture}

The purpose of this section is to highlight that there are contradictory solutions/solutions with tradeoffs with different architectures based on your perspective and concerns. 

An example of this is the tradeoff between energy consumption and mobile gateway bottleneck problem: Some suggest that the mobile gateway should only be used as a device to administer roaming: That the WBAN sensors should send physiological data directly to a boarder router. This proposition solves the problem of the mobile gateway becoming a bottleneck in the WBAN in the case where all traffic have to be routed through the MG. However, this has a tradeoff with energy efficiency: With this organization the node antennas would have to consume more energy in order to send data directly to an access point/boarder router which would typically be located further away than the mobile gateway. This also conflicts with the less discussed topic of interoperability. Should the backend monitoring system not be developed by the same manufacturer as the WBAN nodes, data need to be packaged into interoperable/compliant formats for the information exchange. They might use a reduced format/document structure in order to save bandwidth and processing power. If routing all data through a mobile gateway, this could ensure interoperable data by packaging the data in compliant formats before sending it to the external endpoint. In addition the data would need to be associated with a patient. Such authentication would most likely have to happen through a device with a graphical user interface, such as a touch device acting as a mobile gateway. This also happens to be a less flexible solution: By having a mobile gateway, the owner of the gateway, i.e. the patient could decide weather to send the data to endpoint A, B or maybe only store it locally.

On the same side, the consumer market has recently showed an increased interest in ``smart boarder routers'' or ``hubs'', with Bluetooth Smart and 802.15.4 support in addition to the standard 802.11.X interface. Most notably are the OnHub WiFi router from Google \cite{newRef:60} and Cassia Hub Bluetooth Router \cite{newRef:61}.

% TODO MOVED FROM CHAPT 5 START
We believe the usage of smart devices in health care will increase with time. A future where both medical staff and patients use their existing or hospital provided smart/touch devices as a tools in treatment and everyday medical related tasks, seems probable in a not too distant future. Some forms of this have already been deployed, like AFGA Healthcare's ORBISme! tablet sized EMR-device, being used in Germany today \cite{newRef:271}. This device might just as well be used as a WBAN hub, a mobile gateway connecting WBAN sensors to local area network and external services. The notion of a mobile gateway has been widely adopted and discussed \cite{Movassaghi:2014hi, Mohammed:2014dw, Touati:2015gy, EmilJovanov:2005ty} in earlier research.

Something that have not been widely discussed in research, but that is currently being worked on in the industry is data interoperability between different wireless sensors nodes. There is a difference between consumer technology and medical devices. But with the growing interest for personal health monitoring within the consumer market, this distinction easily becomes blurry. Although not having any diagnostic value, it is not improbable that consumer devices soon will see it's way into a medical setting one way or the other: Just recently was the first reported case of medical professionals using historic data form a personal fitness tracker to make an informed, life saving medical decision \cite{newRef:29}.
% TODO MOVED FROM CHAPT 5 END

% subsection architecture (end)

\subsection{Design alternatives} % (fold)
\label{sub:design_alternatives}

One possible configuration could be wiring up a patient with 10 electrodes, but only having 3 of them activated at a time. Further, one would need to calculate the trade off between doing the ECG analysis onboard the chip versus streaming the data continuously to a WBAN gateway. Either way, the central monitoring station would get notified when abnormalities happened. If the signals were fuzzy, or the doctor demanded higher resolution{1} the remaining 7 electrodes could be activated resulting in a 12-lead ECG. This would match the existing practice of doing low resolution monitoring, and getting details on demand.

% subsection design_alternatives (end)

\subsection{Bluetooth} % (fold)
\label{sub:bluetooth}

We need a section where we discuss the Bluetooth throughput.

Had we investigated the limitations of Bluetooth earlier in the project, we might have - as we first discovered this when implementing the code.

Gateway: In retrospect, we see that aiming for making a flexible solution took much more time to develop, and proved to be generate more edge cases, making the experimentation more prone to errors. Instead we could have hard-coded the devices and endpoints into the gateway, which could have eased the execution of the end-to-end experiment. We made a mistake by prioritizing functionality over reliability in the gateway implementation.

% subsection bluetooth (end)

\subsection{Limitations and Advantages} % (fold)
\label{sub:limitations}

% TODO Network disadvantage - we had no control over the network traffic

On interviews: An alternative approach could have been to target departments higher in the organizational level like purchasing organizations, and do the comparison on the information gathered. This approach however could easily suffer from a mismatch between the general guidelines and the deployed technology as well as long turnover times, meaning a lot of time would have gone by just waiting for an answer from the respective higher level organizational departments.

Temporary list of limitations of our study:

\begin{itemize}

  \item We have not looked into signal nor data compression.

\end{itemize}

During the interview with one of the medical professionals it was mentioned that in his experience, the prevalence of cardiac abnormalities of medical interest might happen during physically straining activities like walking stairs etc. However, today they had no practical way of seeing this correlation. The only place where you can monitor the real-time ECG stream is at the central monitoring station. A great deal of planning and synchronization would have to happen in order to conduct a simple ECG-test on a patient under physically strain: One would have to record the time and place while watching the patient doing the activity, then return to the monitoring central and look at the history in order to get a clear view of how the patient's heart reacted to the activity. A better solution would be ability to watch the ECG on a tablet device while observing the patient doing the activity. An improved version of this would be to record indoor location, a step counter and altitude together with the ECG. This way, the medical professional wouldn't have to consult with neither the patient nor the monitoring central. They could just look at the historic ECG data with activity data overlaid.

% subsection limitations (end)

% section discussion (end)