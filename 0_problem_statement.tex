\noindent
\textbf{Problem Statement} 
\newline 
	
\noindent 
Electronic patient monitoring has been practiced at hospitals for almost half a century. Metrics like ECG, arterial blood pressure, SpO2, respiratory rate and body temperature are measured on every patient in a modern intensive care unit. The consumer industry, the driving force behind many technological advances, have the last couple of years also showed interest in capturing physiological data. With this interest comes new ideas, new solutions and new technology from companies that embed a customer centric focus. The health care has over the years grown into a collaboration of technologically isolated silos, whereas the consumer industry is working towards a ubiquitous reality. In this reality every device and service is connected to each other, and available at any given time. Having the ability to monitor several physiological metrics wirelessly holds potential to improve medical decisions, reduce cables, ease everyday routines, as well as giving ambulatory patients the freedom to move. 
	
	Low energy physiological sensors is a promising technology that will have an increased importance for medicine in the future. These sensors cannot be wired, hence there are many practical (ease of use, privacy, security, compatibility, value, price, safety etc.) and technical (interoperability, minimum delay, maximum throughput, maximum network lifetime, minimum energy consumption) challenges that need to be addressed.
	
	Among different physiological metrics, ECG makes for a reasonable study for two reasons: 
	\begin{itemize}
		\item There are high technical requirements to medical grade ECG. 
		\item ECG is already monitored wirelessly today through telemetry solutions, which makes for a good case study. 
	\end{itemize}

\noindent	
Today's telemetry systems have a short battery lifetime, and the systems and infrastructure are specialized for ECG monitoring. Wireless monitoring today is therefore expensive and offer little flexibility. 

In this master thesis we want to answer the following: Is it possible to create a monitoring solution for wireless ECG, based on available technology and open standards?