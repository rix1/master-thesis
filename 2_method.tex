\section{Method} % (fold)
\label{sec:method}

Our research strategy consist mainly of case studies and experiments.

\begin{itemize}

  \item Our deep dive into ECG monitoring can be viewed as a case study.

  \item We did experiments on the prototype

  \item For data collection we have done both interviews, observations and studied documents.

  \item From the interviews and observations we have only gathered qualitative data (quantitative in terms of number of electrodes?). For data analysis, should I refer to a Discourse Analysis - looking for implicit and hidden meanings? Or what about grounded theory (coding the sources over time)

\end{itemize}



\subsection{Research Strategies and Methods} % (fold)
\label{sub:research_strategies_and_methods}


Like in any projects studying involving people and technology there is an inherent complexity that the researchers have to unravel in order to understand the forces that guide the socio-technical systems in front of them.

When studying these systems it is important to understand why and how an prototype works in it's environment, not only that it works. This is complete field of research in itself, and several approaches to achieving this insight have been proposed \cite{DesignThinkingAndOthers}. My approach was influenced by both the Design Science methodology, my personal experience and IBM's framework for Design Thinking: 

\begin{quote} 
\textit{Observe > Reflect > Make}
\end{quote}

\subsubsection{Interviews} % (fold)
\label{ssub:interviews}

This section will describe interviews, their pros and cons and how they are conducted.

Different stakeholders may have different views and knowledge about various aspects of the challenges with todays monitoring systems. In order to achieve a deeper understanding of the context and difficulties, we decided to approach different stakeholders.

% subsubsection interviews (end)

\subsubsection{Observation} % (fold)
\label{ssub:observation}

This section will describe observations, the abilities it enables and how they are conducted.

% subsubsection observation (end)

\subsubsection{Data Collections and Analysis} % (fold)
\label{ssub:data_collections_and_analysis}

As we will see in Section~\ref{sec:method}, literature review was a major part of this research project in terms of data collection and analysis. A lot of work has been done within the field of WBANs and WSNs, and a systematic approach to gathering and analyzing and extracting knowledge from previous work was an absolute necessity. Literature was also consulted in order to learn about patient monitoring.

However, wireless technology is changing at a rapid pace and certain practical issues are directly linked to the technology in question. This puts a limit on the information and knowledge you can gather from literature. Because of this, we chose interviews as a method for gathering information about the domain and existing practices. A lot of knowledge is embedded in the people and practices of their every day routines. We used both unstructured and semi-structured interviews and with a focus on open questions. This way we were able to develop a understanding of how people viewed their work and how they interpreted the technology that was part of their everyday routines. This also made the research more flexible, and we were able to do interviews on a short notice. This strategy enabled us to talk to people in their natural working environment.

We took note of people's experience with, interpretations of, and reactions to the technical monitoring systems they had at hand. We observed what people said and did, and afterwards reflected over what they did not say or do. This way we were able to search for latent needs in the existing practices. 

Two stakeholders was selected for the survey: Medical engineers and medical practitioners. In order assess the validity of our observations, we chose to conduct interviews in two rounds on two separate hospitals, talking with stakeholders in both and comparing the results. This approach was helpful not only because we got to compare the technical solutions deployed but also the practices around patient monitoring. An alternative approach could have been to target departments higher in the organizational level like purchasing organizations, and do the comparison on the information gathered. This approach however could easily suffer from a mismatch between the general guidelines and the deployed technology as well as long turnover times, meaning a lot of time would have gone by just waiting for an answer from the respective departments.

% subsubsection data_collections_and_analysis (end)

% subsection research_strategies_and_methods (end)

% section method (end)