\section{Research Strategies} % (fold)
\label{sec:method}

The most critical success factors in every research project lies in the researcher's ability to plan, execute an monitor the study. A clear research strategy needs to be clearly defined, revised and followed. This chapter will lay out the research strategy applied in this project, and discuss how we approached the different research stages.

\subsection{Research Methods} % (fold)
\label{sub:research_methods}

Wireless monitoring of physiological metrics is a wide topic covering many similar or related instances of the same problem. In order to reduce the scope we decided to look at only one instance, namely wireless ECG monitoring. Because of this, the project in it self can be considered a case study in ECG. It was important to us that we investigated ECG in depth, as well as studying the practice in it's natural, clinical setting. This is reflected in our data collection methods, which include multiple sources and methods and has a focus on relationships and processes.

Like in any project involving people and technology there is an inherent complexity that the researchers have to unravel in order to understand the forces that guide the socio-technical systems in front of them. We want to avoid the deterministic approach of attributing properties to technology that guarantee the effects and outcome it will have on health care. For this research project it was clear that we needed to approach the practical clinical aspects in addition to the technical aspects of ECG monitoring. Different stakeholders may have different views and knowledge about various aspects of the challenges with todays monitoring systems. In order to achieve a deeper understanding of the context and difficulties, we decided to approach both medical professionals and technical engineers. Because of this, we differentiate our data collection methods in those applied when researching the technical side of these systems, and those applied when exploring the medical domain and current practices.\\
\newline
\noindent
\textbf{Clinical:} In order to understand how and why ECG monitoring is practiced, we applied mainly qualitative techniques. This was done through observations and interviews with medical professionals, and will be discussed further in Section~\ref{ssub:interviews} and Chapter \ref{sec:clinical_context}.\\
\newline
\noindent
\textbf{Technical:} Our strategy for researching the technical requirements to clinical ECG monitoring systems involved understanding both today's existing technical solutions and those proposed in literature. Because this a wide field spanning different domains, several methods for data collection have been utilized. These include interviews, document and literature analysis, and experiments, as discussed further in Section~\ref{sec:experiments} and Chapter~\ref{sec:technology_assessment}.
\\
\newline
\noindent
When studying these systems it is important to understand why and how an prototype works in it's environment, not only that it works. Our overall approach have been influenced by both the Design Science methodology, the research team's personal experiences and IBM's framework for Design Thinking \cite{ibmDesignThinking}. The latter embrace a iterative strategy for understanding peoples contextual needs and deliver outcomes:

\begin{quote} 
Observe > Reflect > Make
\end{quote}
\\
\noindent
This neatly ties together our study of clinical practices with the technical research.


\subsection{Interviews and observations} % (fold)
\label{sub:interviews}

% subsection interviews (end)
This section will describe interviews, how they were conducted and the inherent strengths and weaknesses of doing interviews.

In order assess the validity of our observations and get multiple sources, we decided to conduct interviews at two different Norwegian hospitals. This way we could either record the differences in ECG monitoring practices or confirm that they were the same. Norwegian hospitals are usually organized with an in-house department for technical engineers, which made it easier for us to organize the interviews with different stakeholders.

A lot of knowledge is embedded in the people and practices of their every day routines. An introductory conversation at the first hospital revealed that nurses at both the intensive care units and the cardiac observation unit had most practical experience with the technology. In these units they handle ECG and telemetry solutions on a daily basis. After this quick survey we constructed both unstructured and semi-structured interviews for both hospitals with a focus on open questions. This way we were able to develop a understanding of how people viewed their work and how they interpreted the technology that was part of their everyday routines. This also made the research more flexible, as we were able to do interviews with nurses in their natural working environment on a short notice.

We also arranged for two observation sessions at the second hospital, in order to see for ourself how different aspects of patient monitoring were carried out. In the first session we observed a nurse preparing and conducting a full 12-lead ECG test on a colleague, and in the second session we observed another nurse monitoring 10 remote telemetry patients, and 6 bedside patients in a intensive care unit. During both observations, we got the chance to ask follow-up questions, which further increased our understanding.

Although nurses have most practical experience with the monitoring practices, they do not assess the ECG plots diagnostically. Because of this we also conducted two semi-structured interviews, with a cardiovascular surgeon and an anesthesiologist. These interviews were conducted later in the project and provided clarifications, useful diagnostic insight as well as domain knowledge.

Because this is a highly specialized field, most knowledge is embedded in the workplace. This also holds for the technical knowledge. In order to learn about the technical solutions backing the ECG monitoring we reached out to the clinical engineers at both hospitals. Conducting structured interviews was inexpedient in the beginning, and we began with unstructured interviews at the first hospital. After we had established a basic understanding of the components and solutions, we moved on to conduct semi-structured interviews at the second hospital. This enabled comparison of the technical solutions between the two hostpial.

As mentioned in Chapter~\ref{sec:literature_review}, wireless body area networks and sensor technology are active research areas. In order to gain valuable insight in this development we conducted another unstructured interview with a researcher working with short-range wireless monitoring in critical care at The Intervention Centre at Oslo University Hospital.

All in all we conducted 8 different interview sessions, talking to a total of 10 people. This included 4 clinical engineers, 3 nurses, 2 physicians as well as the clinical researcher from The Intervention Centre. During all interviews we took note of people's experience with, interpretations of, and reactions to the technical monitoring systems and daily routines. We observed what people said and did, and afterwards reflected over what they did not say or do. This way we were able to search for latent needs in the existing practices. The knowledge gained from these interviews are treated in  Section~\ref{sec:clinical_context}.

% subsubsection interviews (end)

\subsection{Experiments} % (fold)
\label{sub:experiments}

% TODO Wirite about how we want to make a protoype 

In order to answer our problem statement and main research question, we wanted to build a working prototype based on available technology and open standards. The prototype was validated through experiments, testing certain critical aspects of ECG based on the knowledge gained throughout the project. These experiments can be found in Subsection~\ref{sub:evaluation}. We wanted to test if these minimum requirements could be upheld, and did not exceed the requirements required for clinical use.

% subsection experiments (end)


\subsection{Document and literature analysis} % (fold)
\label{sub:document_and_literature_analysis}

A lot of work has been done within the field of WBANs and telemetry, and a systematic approach to gathering, analyzing and extracting knowledge from previous work was an absolute necessity. This section will describe how we collected literature and documents from previous research and existing solutions to investigating the topic of WBANs.

Searching for information was a maturing process. Finding the right search terms did take some time, but we found the most relevant results on Google Scholar using a combination of keywords such as \textit{``WBAN, WSN, patient monitoring, BLE, Bluetooth Smart, ambulatory patient monitoring, ECG''}.

Documents were collected from two sources: The Norwegian Directorate for e-Health supplied us with standardization documents and guidelines from Continua Health Alliance. These are documents describing interoperability aspects such as how data should be exchanged between sensors, gateways and end services. We received in total 8 documents in the H.810-813 series, dated 2014. One of the clinical engineers also supplied us with installation and service manuals to the telemetry system they had installed. This were 2 documents and one presentation dated 2007, describing the Philips IntelliVue telemetry system in detail.

% subsection document_and_literature_analysis (end)

% section method (end)