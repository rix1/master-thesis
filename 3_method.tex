\section{Research Strategies} % (fold)
\label{sec:method}

The most critical success factors in every research project lies in the researcher's ability to plan, execute an monitor the study. A clear research strategy needs to be clearly defined, revised and followed. This chapter will lay out the research strategy applied in this project, and discuss how we approached the different research stages.

\subsection{Research Methods} % (fold)
\label{sub:research_methods}

Wireless monitoring of physiological metrics is a wide topic covering many similar or related instances of the same problem. In order to reduce the scope we decided to look at only one instance, namely wireless ECG monitoring. Because of this, the project in it self can be considered a case study in ECG. It was important to us that we investigated ECG in depth, as well as studying the practice in it's natural, clinical setting. This is reflected in our data collection methods, which include multiple sources and methods and has a focus on relationships and processes.

Like in any project involving people and technology there is an inherent complexity that the researchers have to unravel in order to understand the forces that guide the socio-technical systems in front of them. For this research project it was clear that we needed to approach the practical clinical aspects in addition to the technical aspects of ECG monitoring. Because of this, we differentiate our data collection methods in those applied when researching the technical side of these systems, and those applied when exploring the medical domain and current practices.\\
\newline
\noindent
\textbf{Clinical:} In order to understand how and why ECG monitoring is practiced, we applied mainly qualitative techniques. This was done through observations and interviews with medical professionals, and will be discussed further in Section~\ref{ssub:interviews} and Chapter \ref{sec:clinical_context}.\\
\newline
\noindent
\textbf{Technical:} Our strategy for researching the technical requirements to clinical ECG monitoring systems involved understanding both today's existing technical solutions and those proposed in literature. Because this a wide field field, several methods for data collection has been utilized. These include interviews, document and literature analysis, as well as experiments, and is discussed further in Section~\ref{sec:experiments} and Chapter~\ref{sec:technology_assessment}.
\\
\newline
\noindent
When studying these systems it is important to understand why and how an prototype works in it's environment, not only that it works. Our overall approach have been influenced by both the Design Science methodology, the research team's personal experiences and IBM's framework for Design Thinking \cite{ibmDesignThinking}. The latter embrace a iterative strategy for understanding peoples contextual needs and deliver outcomes:

\begin{quote} 
\textit{Observe > Reflect > Make}
\end{quote}
\\
\noindent
This neatly ties together our study of clinical practices with the technical research.


\subsection{Interviews and observations} % (fold)
\label{sub:interviews}

% TODO skrive om hvordan jeg gjorde det på to sykehus for å få flere sources

% subsection interviews (end)
This section will describe interviews, their pros and cons and how they are conducted.

Different stakeholders may have different views and knowledge about various aspects of the challenges with todays monitoring systems. In order to achieve a deeper understanding of the context and difficulties, we decided to approach different stakeholders.

% subsubsection interviews (end)

This section will describe observations, the abilities it enables and how they are conducted.

\subsection{Experiments} % (fold)
\label{sub:experiments}

% TODO Wirite about how we want to make a protoype 

% subsection experiments (end)


\subsubsection{Data Collections and Analysis} % (fold)
\label{ssub:data_collections_and_analysis}

As we will see in Section~\ref{sec:method}, literature review was a major part of this research project in terms of data collection and analysis. A lot of work has been done within the field of WBANs and WSNs, and a systematic approach to gathering and analyzing and extracting knowledge from previous work was an absolute necessity. Literature was also consulted in order to learn about patient monitoring.

However, wireless technology is changing at a rapid pace and certain practical issues are directly linked to the technology in question. This puts a limit on the information and knowledge you can gather from literature. Because of this, we chose interviews as a method for gathering information about the domain and existing practices. A lot of knowledge is embedded in the people and practices of their every day routines. We used both unstructured and semi-structured interviews and with a focus on open questions. This way we were able to develop a understanding of how people viewed their work and how they interpreted the technology that was part of their everyday routines. This also made the research more flexible, and we were able to do interviews on a short notice. This strategy enabled us to talk to people in their natural working environment.

We took note of people's experience with, interpretations of, and reactions to the technical monitoring systems they had at hand. We observed what people said and did, and afterwards reflected over what they did not say or do. This way we were able to search for latent needs in the existing practices. 

Two stakeholders was selected for the survey: Medical engineers and medical practitioners. In order assess the validity of our observations, we chose to conduct interviews in two rounds on two separate hospitals, talking with stakeholders in both and comparing the results. This approach was helpful not only because we got to compare the technical solutions deployed but also the practices around patient monitoring. An alternative approach could have been to target departments higher in the organizational level like purchasing organizations, and do the comparison on the information gathered. This approach however could easily suffer from a mismatch between the general guidelines and the deployed technology as well as long turnover times, meaning a lot of time would have gone by just waiting for an answer from the respective departments.

% subsubsection data_collections_and_analysis (end)

% subsection research_strategies_and_methods (end)

% section method (end)