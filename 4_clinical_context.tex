\section{Clinical Context} % (fold)
\label{sec:clinical_context}

This section describes the requirements for the prototype. First we outline the prototype, then an introduction of the stakeholders and their roles follow, before elaborating on non-functional requirements. Lastly, a comparison of the different stakeholders and their concerns are conducted. 

Below is a description of the three most important stakeholders.

\begin{itemize}

  \item[Patients] The patients themselves are the main source of physiological data. The patients are the ones in direct contact with the sensory equipment, and will act as the wearer of the  WBANs. It is important to remember that these are people admitted to the hospital with a broad span of medical conditions. They might be in pain, unconscious, or mentally exhausted. One must take these considerations into the design and requirement specification.

  \item[Physicians] Physicians and clinicians are the medical staff that are responsible for care and treatment of patients, as well as operating their respective wards and departments. They are the ones that will install the sensors on the patient.

  \item[Technical engineer] Technical engineer is defined as the medical and biomedical maintenance engineer responsible for installing and maintaining the technical equipment used for medical purposes at hospitals. Among other things, they are responsible for doing technical maintenance  on the central heart monitoring system, which is as close as we come to an established way of monitoring vital signs of multiple patients in hospitals today.

\end{itemize}


\subsection{Intended use} % (fold)
\label{sub:intended_use}

I paid several visits to both St. Olavs and Vestre Viken HF in order to observe and talk to experts in the domain of patient monitoring. Karl Øyri made it clear that I had to consider the *intended use* for my prototype. However convenient, monitoring ambulatory patients is merely an action following an an intention, not the goal in itself. As mentioned, many before me have taken the rather deterministic approach of looking at the technology and claiming it will revolutionize health care because of its properties and possibilities. Therefore we must ask the question, for what patient profile are we solving the problem? What use cases? What is the intended use?

At Vestre Viken a centralized monitoring station similar to the one I visited at St. Olavs was located at the intensive care unit. In addition to patients located in the ICU, they also monitored patients in other parts of the hospital connected via telemetry. Both the local and the remote patients were connected to a 5 lead (EASI) ECG setup. If there was something suspicious with the readings on the screen, both the ICU and the unit with the remote patients had several standalone 12-lead ECG units that could be connected to the patient on request. Through a demonstration, I was shown how a patient was connected to the standalone ECG device with 10 electrodes in order to get the full 12-lead resolution. This device was mobile (on wheels) and had the form factor of a small laptop computer. It was standalone in the sense that it's only output was a printed exert from the examination as shown in Figure~\ref{fig:pappa_ecg} from a built in printer. We also visited two other medical units at the same hospital that did ECG monitoring. Both of which used a 3-lead setup. This stands to show that at Vestre Viken, 3 and 5-lead setups are most used in continuous monitoring. The full 12-lead was only used in short examinations on demand. This practice was also confirmed at the Clinic of Cardiology at St. Olavs.

% TODO Insert pappa_ecg

Based on these observations, the intended use of the prototype should be similar to the role played by telemetry systems today. These systems are used for monitoring patients that are expected to experience, or are vulnerable to, heart defects either pre or post surgery, or during treatment. See table Table~\ref{tab:table_heart_irregularities} for a description of the most common irregularities and how they are identified. Because irregularities can happen at any time of the day, continuous monitoring is a fundamental requirement to the prototype.

% TODO Insert table_heart_irregularities

% TODO Insert footnote
\footnote{Through conversations with a medical professional it became clear that that the prevalence of heart abnormalities of medical interest was likely to happen during physically straining activities like walking stairs etc. However, this could also be happening while resting, as confirmed by an patient who experienced irregular heart beats (cardiac arrhythmia) while sleeping}


% subsection intended_use (end)

\subsection{Existing Solutions} % (fold)
\label{sub:existing_solutions}

What can we learn from the systems employed today?

Interviews with medical engineers at both hospitals revealed that both had a telemetry system from Philips installed. In both cases the network infrastructure and equipment was produced by Philips, but was delivered and maintained by different partners (Vingmed AS at Vestre Viken and HEMIT at St. Olavs). In this section I will describe the IntelliVue telemetry system from Philips, which is deployed at Norwegian hospitals today. The studying these systems were conducted in order to gain insight in the practice of wireless patient monitoring today, as well as giving us an indication of how prepared hospitals are for WBANs from a infrastructural point of view.

\subsubsection{System Architecture} % (fold)
\label{ssub:system_architecture}

The Philips IntelliVue telemetry system is a complete end-to-end system enabling wireless monitoring of ECG data. For the medical practitioners the system is composed of two components: the TRx4841/51A wireless transceivers and the IntelliVue Information Centre. The TRx transceivers are devices for capturing ECG and SpO2 on adult and pediatric patients, and are worn by the patients. Powered by two AA batteries they  capture and stream a 3 or 5 lead ECG to a custom 802.11 type access point (AP) operating in the 2,4 GHz spectrum. These APs are installed in addition to, and independent of existing public or private WiFi access points. Because Philips provide all hardware in this ``closed loop'' system, the telemetry system does not have to confine to the interoperability requirements it would needed if the system were to allow for proprietary sensors and measuring devices. The wireless interface does not strictly follow the 802.11 specification, and the system offer a ``Smart Hopping'' technology in order to reduce collisions on the already crowded 2,4 GHz network band. This is made possible by synchronizing all access points through a dedicated sync unit. Another noteworthy network feature is the ``make before break'' method that improves the roaming abilities. The TRx transceivers will look for new, stronger access points when they detect a decay in signal strength with the current AP. The system will make sure to connect transceivers to the new (stronger) AP before breaking with the previous one, enabling a more reliable handoff between access points. The network coverage on both hospitals were 

As mentioned, this being a end to end system the IntelliVue patient monitoring suite includes transceivers, bedside monitors, access points, routers, gateways, network switches, uninterruptible power supplies (UPS), processing units and databases and a optional printer Figure~\ref{fig:intellivue_telemetry_architecture}.

% TODO Insert figure

Offering every piece in this architecture enables Philips and its local distributers/providers to guarantee good QoS metrics which is essential for wireless patient monitoring applications. However, installing a complete system like this comes with a substantial cost, and ties you down to just one provider for an extended period of time. The IntelliVue Telemetry system at St. Olavs was installed in 2009, and is expected to have a lifespan of at least 10 years. This reduces the flexibility and increases the need for interoperability and good integrations with other systems. An example of this lack of flexibility can be found in the installation of the CorPulse ECG device from Alere \cite{alere}. This is a device installed in emergency vehicles enabling pre-hospital ECG to be sent via a SIM card and the 3G network. This is often critical information that enable the medical staff to make informed decisions before the patient arrives at the hospital. Because this is a different system from a different provider, the hospitals we visited had installed a stand alone computer next to the information center system from Philips, that displayed the pre-hospital ECG from the Alere device. This is suboptimal not only in technical terms and with regards to maintenance, but also from a user experience point of view. The lack of integrations become yet another technical obstacle in the everyday life of medical practitioners, and ways around the barriers are found. As seen (although censored) in Figure~\ref{fig:bilde_av_alere_ecg}, credentials are often shared as a result of this, rendering the security measure useless. Based on these observations, we believe interoperable devices and services will be crucial for the success of a ubiquitous health care.

% subsubsection system_architecture (end)

\subsection{The Transceivers} % (fold)
\label{sub:the_transceivers}

Although the IntelliVue wireless transceivers provide ambulatory and bedside patient monitoring, there's a catch with increasing the mobility other than the movements themselves. Usually when doing a detailed 12-lead ECG, medical practitioners place the 4 extremity electrodes at the wrists ankles. In telemetry these electrodes are often moved to shoulders and hips in order to increase mobility. However, because of the large muscle mass located in these areas, the recorded ECG signal is prone to more muscle noise. We observed this ourself while studying a 5-lead telemetry reading at one of the hospitals. Accurate, unobtrusive and noise resilient sensors like the ones in \cite{ChulsungPark:2006tf, Anonymous:FtVb5yQr} will be an absolute necessity in persuasive patient monitoring. 

\cite{}
(...)

Unlike multimedia streams, a physiological data stream cannot be buffered to mask jitter and latency on the receiving end. End to end latency will therefore be experienced as gaps in the monitoring stream, thereby invalidating the purpose of the system.

% TODO Insert this somewhere

\footnote{It's worth noting however that a decision support system based on data mining would usually not have the same real-time requirements and would not suffer from potential latency}

% subsection the_transceivers (end)

\subsubsection{The Information Centre} % (fold)
\label{ssub:the_information_centre}

The information center can provide a web view for accessing the patient data from outside the monitoring central. According to the service manuals they claim that the information displayed in the web view should be near-real time with little delay, but recommend users to always use the bedside monitor or the information center for real-time monitoring. We were unable to find a accurate metric for the end-to-end latency in IntelliVue's documentation. However, because the providers have full control over the whole infrastructure and the distances at a hospital are relatively small, it is reasonable to assume this delay is less than 2 seconds from the measurement to the information center.

An alternative approach to getting information about acceptable transmission delay would be to measure the actual network traffic in the systems deployed today. However, we were unsuccessful in getting in touch with the network administrator at either hospitals.

Information on transmission delay is also scarce in the literature we reviewed. Within the field of WBAN very few address this issue, and and we had to look into Telemedicine in order to find any suggestions to end to end latency at all. Through clinical trails, Alesanco and Garcia suggest 4 seconds as the maximum acceptable delay for a reliable real-time ECG transmission. Their research however is concerned with telemedicine over 2G/3G networks, and the intended use and QoS are different. This being the only exact metric we could relate to, we chose to use this as our baseline for network delay.

% subsubsection the_information_centre (end)

% subsection existing_solutions (end)

\subsection{Requirements} % (fold)
\label{sub:requirements}

We are going to wrap up Section~\ref{sec:literature_review} and ~\ref{sec:interviews} up with an overview of the functional and non functional requirements for a modern monitoring system.

\subsubsection{Non functional Requirements} % (fold)
\label{ssub:non_functional_requirements}

(I should limit this to my scope)

% subsubsection non_functional_requirements (end)

\subsubsection{Functional Requirements} % (fold)
\label{ssub:functional_requirements}

What requirements does physicians demand from physiological sensors? In this section we discuss different requirements such as timeliness, ECG resolution and sample rate. (This could be moved to literature review)

% subsubsection functional_requirements (end)

% subsection requirements (end)

% section clinical_context (end)